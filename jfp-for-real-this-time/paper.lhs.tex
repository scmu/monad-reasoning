\documentclass[acmsmall,review,anonymous]{acmart}

%\setmonofont{JetBrains Mono}[
%  Scale=MatchLowercase
%]


%%
%% \BibTeX command to typeset BibTeX logo in the docs
\AtBeginDocument{%
  \providecommand\BibTeX{{%
    \normalfont B\kern-0.5em{\scshape i\kern-0.25em b}\kern-0.8em\TeX}}}

%% Rights management information.  This information is sent to you
%% when you complete the rights form.  These commands have SAMPLE
%% values in them; it is your responsibility as an author to replace
%% the commands and values with those provided to you when you
%% complete the rights form.
\setcopyright{acmcopyright}
\copyrightyear{2021}
\acmYear{2021}
\acmDOI{xxx}


%%
%% These commands are for a JOURNAL article.
%\acmJournal{JACM}
%\acmVolume{37}
%\acmNumber{4}
%\acmArticle{111}
%\acmMonth{8}

%% These commands are for a PROCEEDINGS abstract or paper.
% \acmConference[ICFP '21]{ICFP '21}{August 22--27, 2021}{Virtual}
%\acmBooktitle{ICFP '21}
%\acmPrice{15.00}
%\acmISBN{978-1-4503-XXXX-X/18/06}

%%
%% Submission ID.
%% Use this when submitting an article to a sponsored event. You'll
%% receive a unique submission ID from the organizers
%% of the event, and this ID should be used as the parameter to this command.
%%\acmSubmissionID{123-A56-BU3}

%%
%% The majority of ACM publications use numbered citations and
%% references.  The command \citestyle{authoryear} switches to the
%% "author year" style.
%%
%% If you are preparing content for an event
%% sponsored by ACM SIGGRAPH, you must use the "author year" style of
%% citations and references.
%% Uncommenting
%% the next command will enable that style.
\citestyle{acmauthoryear}

\newcommand{\todo}[1]{\textcolor{red}{\underline{TODO:} #1}}
\newcommand{\birthe}[1]{\textcolor{blue}{\textbf{Birthe:} #1}}
% \newcommand{\wenhao}[1]{\textcolor{violet!50!blue}{\textbf{Wenhao:} #1}}
\newcommand{\wenhao}[1]{{\par\noindent\color{violet!50!blue} \framebox{\parbox{\dimexpr\linewidth-2\fboxsep-2\fboxrule}{\textbf{Wenhao:} #1}}}}
\newcommand{\tom}[1]{\textcolor{purple}{\textbf{Tom:} #1}}

\usepackage{stmaryrd}
\usepackage{xcolor}
% \usepackage[capitalise]{cleveref}
\usepackage{url}
\usepackage{listings}
\usepackage{booktabs}
\usepackage{subcaption}
\usepackage{soul}
\usepackage{scalerel}
\usepackage{mathtools}
\usepackage{quiver}
\lstset{language=C,basicstyle=\ttfamily}

\setlength{\fboxsep}{5pt}

% array stuff
\newcommand{\ba}{\begin{array}}
\newcommand{\ea}{\end{array}}

\newcommand{\bl}{\ba[t]{@{}l@{}}}
\newcommand{\el}{\ea}

% colorful proofs
\newcommand{\refa}[1]{{\color{red}    {#1}\ifthenelse{\equal{#1}{}}{}{\,}(1)}}
\newcommand{\refb}[1]{{\color{blue}   {#1}\ifthenelse{\equal{#1}{}}{}{\,}(2)}}
\newcommand{\refc}[1]{{\color{violet} {#1}\ifthenelse{\equal{#1}{}}{}{\,}(3)}}
\newcommand{\refd}[1]{{\color{purple} {#1}\ifthenelse{\equal{#1}{}}{}{\,}(4)}}
\newcommand{\refe}[1]{{\color{cyan}   {#1}\ifthenelse{\equal{#1}{}}{}{\,}(5)}}
\newcommand{\reff}[1]{{\color{magenta}{#1}\ifthenelse{\equal{#1}{}}{}{\,}(6)}}
\newcommand{\refg}[1]{{\color{brown}  {#1}\ifthenelse{\equal{#1}{}}{}{\,}(7)}}
\newcommand{\refh}[1]{{\color{orange} {#1}\ifthenelse{\equal{#1}{}}{}{\,}(8)}}
\newcommand{\refs}[1]{{\color{gray} {#1}\ifthenelse{\equal{#1}{}}{}{\,}($\star$)}}
\let\Bbbk\undefined
%include polycode.fmt
%include forall.fmt
%include fmts/symbols.fmt

% add margin before code
\usepackage{enumitem}
\newlist{qwq}{itemize}{1}
\setlist[qwq]{label={}, nosep}
\newcommand{\indentbegin}{\begin{qwq} \item}
\newcommand{\indentend}{\end{qwq}}
%subst code a = "\indentbegin \begin{hscode}\SaveRestoreHook'n" a "\ColumnHook'n\end{hscode}\resethooks'n\indentend "

%%
%% end of the preamble, start of the body of the document source.
\begin{document}

%%
%% The "title" command has an optional parameter,
%% allowing the author to define a "short title" to be used in page headers.
\title{From High to Low: Simulating Nondeterminism and State with State}

%% ALTERNATIVE TITLES
% Simulating Nondeterminism and State
% Simulating Nondeterminism and State with State
% Do it at a Lower Level
% Simulating Higher-Level Effects with Lower-Level Effects
% From High to Low: a Simulation
% From High to Low: Simulating Nondeterminism and State with State
% Nondeterminism + State = State

%%
%% The "author" command and its associated commands are used to define
%% the authors and their affiliations.
%% Of note is the shared affiliation of the first two authors, and the
%% "authornote" and "authornotemark" commands
%% used to denote shared contribution to the research.
% \author{name}
% %\authornote{Both authors contributed equally to this research.}
% \email{email}
% \orcid{orcid}
% \affiliation{%
%   \institution{KU Leuven}
%   \country{Belgium}
% }

%%
%% By default, the full list of authors will be used in the page
%% headers. Often, this list is too long, and will overlap
%% other information printed in the page headers. This command allows
%% the author to define a more concise list
%% of authors' names for this purpose.
%\renewcommand{\shortauthors}{Trovato and Tobin, et al.}

%%
%% The abstract is a short summary of the work to be presented in the
%% article.
\begin{abstract}
  Just as of programming languages, one can think of some effects as being more
  low-level than others. Indeed, particular effects allow a more fine-grained
  control over program execution and resources, while others offer a higher level
  of abstraction. It is usually desirable to
  write programs using higher-level effects, and at the same time still
  benefit from the optimizations that lower-level effects afford.

  This paper studies how higher-level effects can be simulated in terms of low-level effects.
  In particular, our contribution focusses at the possible interactions between
  state and nondeterminism.
  We model these interactions using a single state effect and prove all intermediate
  steps and results correct.
  These proofs use equational reasoning techniques and stand out
  from other proofs because of the presence of algebraic effects. \birthe{??}
  We distinguish between local-state and global-state semantics
  and transform the higher-level nondeterminism to state. We demonstrate
  possible optimizations such as efficient backtracking and mutable state.
  We illustrate our simulations on the well-known n-queens example and
  prove all simulation functions correct using equational reasoning techniques.
\end{abstract}

%%
%% The code below is generated by the tool at http://dl.acm.org/ccs.cfm.
%% Please copy and paste the code instead of the example below.
%%
% \begin{CCSXML}
% 	<ccs2012>
% 	<concept>
% 	<concept_id>10011007.10011006.10011008.10011009.10011012</concept_id>
% 	<concept_desc>Software and its engineering~Functional languages</concept_desc>
% 	<concept_significance>500</concept_significance>
% 	</concept>
% 	<concept>
% 	<concept_id>10011007.10011006.10011008.10011024.10011027</concept_id>
% 	<concept_desc>Software and its engineering~Control structures</concept_desc>
% 	<concept_significance>500</concept_significance>
% 	</concept>
% 	</ccs2012>
% \end{CCSXML}

% \ccsdesc[500]{Software and its engineering~Functional languages}
% \ccsdesc[500]{Software and its engineering~Control structures}

%%
%% Keywords. The author(s) should pick words that accurately describe
%% the work being presented. Separate the keywords with commas.
\keywords{effects, effect handlers, modularity, nondeterminism,
          state, equational reasoning, monads}

\maketitle

%include sections/introduction.lhs

%include sections/Background.lhs

%include sections/Overview.lhs

%include sections/LocalGlobal.lhs

%include sections/NondetState.lhs

%include sections/combination.lhs

%include sections/MutableState.lhs

% %include sections/optimizations.lhs

%include sections/related_work.lhs

%include sections/conclusion.lhs

%include sections/BenchmarkTable.lhs

%%
%% The acknowledgments section is defined using the "acks" environment
%% (and NOT an unnumbered section). This ensures the proper
%% identification of the section in the article metadata, and the
%% consistent spelling of the heading.
%\begin{acks}
%To Robert, for the bagels and explaining CMYK and color spaces.
%\end{acks}

%%
%% The next two lines define the bibliography style to be used, and
%% the bibliography file.
\bibliographystyle{ACM-Reference-Format}
\bibliography{bibliography}


\clearpage

\appendix

%%
%% If your work has an appendix, this is the place to put it.
%\appendix

%include appendices/local_law.lhs

%include appendices/initiality_nd.lhs

% %include appendices/initiality_state.lhs

%include appendices/nondet_state_sim.lhs

%include appendices/local_global_sim.lhs

%include appendices/states_state_sim.lhs

%include appendices/final_simulate.lhs

\end{document}
\endinput
%%
%% End of file `sample-acmsmall.tex'.
