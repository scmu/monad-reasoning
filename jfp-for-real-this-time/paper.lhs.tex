\documentclass{jfp}

 \DeclareTextCommandDefault{\nobreakspace}{\leavevmode\nobreak\ } 
\newcommand{\todo}[1]{\textcolor{red}{\underline{TODO:} #1}}
\newcommand{\birthe}[1]{\textcolor{blue}{\textbf{Birthe:} #1}}

\usepackage{stmaryrd}
\usepackage{xcolor}
\usepackage[capitalise]{cleveref}
\usepackage{listings}
\usepackage{booktabs}
\usepackage{subcaption}
\usepackage{soul}
\usepackage{scalerel}
\usepackage{mathtools}
\usepackage{quiver}
\lstset{language=C,basicstyle=\ttfamily}
\let\Bbbk\undefined
%include polycode.fmt
%include forall.fmt
%include fmts/symbols.fmt

% add margin before code
\usepackage{enumitem}
\newlist{qwq}{itemize}{1}
\setlist[qwq]{label={}, nosep, leftmargin=1em}
\newcommand{\indentbegin}{\begin{qwq} \item}
\newcommand{\indentend}{\end{qwq}}
%subst code a = "\indentbegin \begin{hscode}\SaveRestoreHook'n" a "\ColumnHook'n\end{hscode}\resethooks'n\indentend "
% NOTE: The above line is not a comment. It is a lhs command to insert indentation for code block.

\newtheorem{theorem}{Theorem}
\newtheorem{lemma}{Lemma}

\begin{document}

\journaltitle{JFP}
\cpr{Cambridge University Press}
\doival{10.1017/xxxxx}

% \lefttitle{}
% \righttitle{Journal of Functional Programming}

\totalpg{\pageref{lastpage01}}
\jnlDoiYr{2022}

\title{From High to Low: Simulating Nondeterminism and State with State}

\begin{authgrp}
\author{Wenhao Tang}
\affiliation{University of Edinburgh \\
        (\email{wenhao.tang@@ed.ac.uk})}
\end{authgrp}
\begin{authgrp}
\author{Tom Schrijvers}
\affiliation{KU Leuven Department of Computer Science \\
        (\email{tom.schrijvers@@kuleuven.be})}
\end{authgrp}

\begin{abstract}
  Just as of programming languages, one can think of some effects as being more
  low-level than others. Indeed, particular effects allow a more fine-grained
  control over program execution and resources, while others offer a higher level
  of abstraction. It is usually desirable to
  write programs using higher-level effects, and at the same time still
  benefit from the optimizations that lower-level effects afford.

  This paper studies how higher-level effects can be simulated in terms of low-level effects.
  In particular, our contribution focusses at the possible interactions between
  state and nondeterminism.
  We model these interactions using a single state effect and prove all intermediate
  steps and results correct.
  These proofs use equational reasoning techniques and stand out
  from other proofs because of the presence of algebraic effects. \birthe{??}
  We distinguish between local-state and global-state semantics
  and transform the higher-level nondeterminism to state. We demonstrate
  possible optimizations such as efficient backtracking and mutable state.
  We illustrate our simulations on the well-known n-queens example and
  prove all simulation functions correct using equational reasoning techniques.
\end{abstract}

\maketitle

%include sections/introduction.lhs

%include sections/Background.lhs

%include sections/Overview.lhs

%include sections/LocalGlobal.lhs

%include sections/NondetState.lhs

%include sections/Combination.lhs

%include sections/MutableState.lhs

% %include sections/optimizations.lhs

%include sections/related_work.lhs

%include sections/conclusion.lhs

\subsection*{Conflicts of Interest}

None.

%%
%% The acknowledgments section is defined using the "acks" environment
%% (and NOT an unnumbered section). This ensures the proper
%% identification of the section in the article metadata, and the
%% consistent spelling of the heading.
%\begin{acks}
%To Robert, for the bagels and explaining CMYK and color spaces.
%\end{acks}

%%
%% The next two lines define the bibliography style to be used, and
%% the bibliography file.
\bibliographystyle{jfplike}
\bibliography{bibliography}


\clearpage

\appendix

%%
%% If your work has an appendix, this is the place to put it.
%\appendix

%include appendices/local_law.lhs

%include appendices/initiality_nd.lhs

% %include appendices/initiality_state.lhs

%include appendices/nondet_state_sim.lhs

%include appendices/local_global_sim.lhs

%include appendices/states_state_sim.lhs

%include appendices/final_simulate.lhs

%include appendices/trail_stack_sim.lhs

%include appendices/additional_equations.lhs

\end{document}
